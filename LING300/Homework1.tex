\documentclass[11pt,notitlepage]{article}
\usepackage{graphicx}
\usepackage{amsmath}            % adds more math symbols
\usepackage{amssymb}
\usepackage[usenames,dvipsnames]{color}
\usepackage{multicol}
\usepackage{listings}
\title{LING-300: Homework 1}
\author{Leo Przybylski\\
\texttt{przybyls@arizona.edu}}

\newcommand{\question}[2]{\textbf{#1.} #2}
\newcommand{\subquestion}[2]{\par\hspace{0.5cm} \textbf{#1)} #2}
\newenvironment{answer}{\endpar%

}

    % Give wider margins; gives more text per page.

\setlength{\topmargin}{0.00in}
\setlength{\textheight}{8.75in}
\setlength{\textwidth}{6.625in}
\setlength{\oddsidemargin}{0.0in}
\setlength{\evensidemargin}{0.0in}

\setlength{\parindent}{0.0cm}	% Don't indent the paragraphs
%\setlength{\parskip}{0.4cm}	% distance between paragraphs

\definecolor{ubergray}{RGB}{245,245,245}
\begin{document}
  \maketitle
  {\setlength{\baselineskip}%
           {0.0\baselineskip}
  \section*{Carnie Chapter 1}
  \hrulefill \par}

\question{3}{Learning vs. Acquisition}

\begin{description}
\item[Other than language are there other things we acquire?]
  Growth is not something we control or think about
  consciously. Rather, it is done subconciously handled by internal
  systems. Sight is not something we think about. It is something we
  acquired. This can be said about any of our senses. Walking has been
  described as something acquired, but related to walking is
  balance. This includes balance while sitting, lying down, walking,
  or riding a bike. The cliche ``like riding a bike'' is actually relating to
  balance being acquired and something that is never lost.

\item[What other things do we learn?] You can train your body and mind
  through repetition. For example, one can memorize their phone number
  by using it repetitiously. Also, lines to a script can be memorized
  by rehearsing or resiting them repetitiously. In the same way,
  memory in general can be exercised. Physical motion is largely
  acquired, but corrected physical motion is learned. For example,
  anyone can swing a golf club if they watch someone do it. It's
  acquired. To swing a golf club in proper form takes a lot of
  practice and training. Eventually, the body learns this through
  muscle memory. Similarly, even though language is acquired. Speaking
  correctly is learned. Anyone has the capacity for language, but a
  lisp can occur anywhere.

\item[What about walking? Or reading? Or sexual identity?] Carnie
 describes walking as something acquired. I agree with this; however,
 it is dependent upon having legs. I am unsure if someone born without
 legs would have acquired the knowledge of walking. Reading falls into
 language. Likewise to walking, it is dependent upon physical
 attributes. Being able to see, hear, or touch are requisites for
 reading and probably also for language. I think sexual identity also
 acquired and connected to gender which is a physical
 attribute. Generative grammars is described by Carnie more of a study
 involving psychology than a cognitive science. I would have to agree
 with that considering walking, reading, or sexual identity. These are
 acquired and even though they involve neurology, they are more
 psychological.
  
\end{description}

\question{4}{Universals}

\begin{description}
  \item[How might you account for the existence of universals (see
    definition above) across languages?] I would take the stance that
    language is not productive afterall. One of the arguments for
    universals across languages is that universals are
    innate. Universals are not innate; therefore, language is not
    productive. I have a hypothesis that intuition is not innate or
    subconscious, but rather a consensus of rules based on repetitive
    communication. Rules are intuitive, but are also
    learned and acquired; therefore, even if a sentence has never been
    heard before, it can still be created by consciously applying the rules.  
\end{description}

\question{6}{Levels of Adequacy}

\subquestion{a}{Juan Martínez has been working with speakers of
  Chicano Eng- lish in the barrios of Los Angeles. He has been looking
  both at corpora (rap music, recorded snatches of speech) and working
  with adult native speakers.}
\begin{quote} Juan uses descriptively adequate level of adequacy
  because he observes both the corpora and the native speakers.
\end{quote}

\subquestion{b}{Fredrike Schwarz has been looking at the structure of sentences in eleventh-century Welsh poems. She has been working at the national archives of Wales in Cardiff.}
\begin{quote}
Fredrike uses observationally adequate grammar because she only
observes the corpus.
\end{quote}

\subquestion{c}{Boris Dimitrov has been working with adults and corpora on the formation of questions in Rhodopian Bulgarian. He is also conducting a longitudinal study of some two-year-old children learning the language to test his hypotheses.}
\begin{quote}
  Boris uses explanatorily adequate grammar because he not only
  observes corpora and native speakers, but also how children acquire language.
\end{quote}

\newpage
  {\setlength{\baselineskip}%
           {0.0\baselineskip}
  \section*{Notes and Instructor Comments}
  \hrulefill \par}
\end{document}

\documentclass[11pt,notitlepage]{article}
\usepackage{graphicx}
\usepackage{amsmath}            % adds more math symbols
\usepackage{amssymb}
\usepackage[usenames,dvipsnames]{color}
\usepackage{multicol}
\usepackage{listings}
\title{LING-300: Homework 4}
\author{Leo Przybylski\\
\texttt{przybyls@arizona.edu}}

\newcommand{\question}[2]{\textbf{#1.} #2}
\newcommand{\subquestion}[2]{\par\hspace{0.5cm} \textbf{#1)} #2}
\newenvironment{answer}{\endpar%

}

    % Give wider margins; gives more text per page.

\setlength{\topmargin}{0.00in}
\setlength{\textheight}{8.75in}
\setlength{\textwidth}{6.625in}
\setlength{\oddsidemargin}{0.0in}
\setlength{\evensidemargin}{0.0in}

\setlength{\parindent}{0.0cm}	% Don't indent the paragraphs
%\setlength{\parskip}{0.4cm}	% distance between paragraphs

\definecolor{ubergray}{RGB}{245,245,245}
\begin{document}
  \maketitle
  {\setlength{\baselineskip}%
           {0.0\baselineskip}
  \section*{Carnie Chapter 3}
  \hrulefill \par}
\question{4}{ENGLISH}
\begin{quote}
[Application of Skills and Knowledge; Basic to Advanced]
Draw phrase structure trees and bracketed diagrams for each of the
follow- ing sentences, indicate all the categories (phrase (e.g., NP)
and word level (e.g., N)) on the tree. Use the rules given above in
the “Ideas” summary of this chapter. Be careful that items that modify
one another are part of the same constituent. Treat words like can,
should, might, was, as instances of the category T (tense). (Sentences
d–h are from Sheila Dooley.)
\end{quote}

\subquestion{e}{Every cat always knows the location of her favorite catnip toy.}

\begin{quote}
[$_{TP}$[$_{NP}$Every cat] [[$_{AdvP}$always] knows [$_{NP}$the location] [$_{PP}$of [$_{NP}$her [$_{AdjP}$favorite] [$_{AdjP}$catnip] toy]]]].

\includegraphics[width=\textwidth]{Diagrams/3_4_e.png}
\end{quote}

\subquestion{f}{The cat put her catnip toy on the plastic mat.}

\begin{quote}
[$_{TP}$[$_{NP}$The cat] [$_{VP}$put [$_{NP}$her [$_{AdjP}$catnip] toy] [$_{PP}$on [$_{NP}$the plastic mat]]]].

\includegraphics[width=\textwidth]{Diagrams/3_4_f.png}
\end{quote}

\subquestion{g}{The very young child walked from school to the store.}

\begin{quote}
[$_{TP}$[The [$_{AdjP}$very [$_{AdjP}$young]] child] [$_{VP}$walked [$_{PP}$from [$_{NP}$school]] [$_{PP}$to [$_{NP}$the store]]]].

\includegraphics[width=\textwidth]{Diagrams/3_4_g.png}
\end{quote}

\subquestion{h}{John paid a dollar for a head of lettuce.}

\begin{quote}
[$_{TP}$[$_{NP}$John] [$_{VP}$paid [$_{NP}$a dollar] [$_{PP}$for [$_{NP}$a head [$_{PP}$of [$_{NP}$lettuce]]]]]].

\includegraphics{Diagrams/3_4_h.png}
\end{quote}

\subquestion{i}{Teenagers drive rather quickly.}

\begin{quote}
[$_{TP}$[$_{NP}$Teenagers] [$_{VP}$drive [$_{AdvP}$rather [$_{AdvP}$quickly]]]].

\includegraphics{Diagrams/3_4_i.png}
\end{quote}

\subquestion{j}{A clever magician with the right equipment can fool the audience
easily.}

\begin{quote}
[$_{TP}$[$_{NP}$A clever magician] [$_{PP}$with [$_{NP}$the [$_{AdjP}$right] equipment]] [$_{VP}$[$_{AdvP}$can] fool [$_{NP}$the audience] [$_{AdvP}$easily]]]].

\includegraphics[width=\textwidth]{Diagrams/3_4_j.png}
\end{quote}


\question{8}{AMBIGUITY}
\begin{quote}
[Application of Knowledge and Skills; Basic to Intermediate]
The following English sentences are all ambiguous. Provide a
paraphrase (a sentence with roughly the same meaning) for each of the
possible meanings, and then draw (two) trees of the original sentence
that distinguish the two meanings. Be careful not to draw the tree of
the paraphrase. Your two trees should be different from one another,
where the difference reflects which elements modify what. (For
sentence (b) ignore the issue of capitalization.) Sentences (c), (d),
(e), and (f) are ambiguous newspaper headlines taken from
http://www.fun\-with\-words.com/ambiguous\_headlines.html). You may need
to assume that old and seven can function as adverbs.
\end{quote}

\subquestion{a}{John said Mary went to the store quickly.}

\begin{quote}John quickly said Mary went to the store.\end{quote}

    \includegraphics{Diagrams/3_8_a1.png}

\begin{quote}John said Mary quickly went to the store.\end{quote}

    \includegraphics{Diagrams/3_8_a2.png}

\subquestion{a}{I discovered an old English poem.}

\begin{quote}I discovered an Old English style poem.\end{quote}

 \includegraphics{Diagrams/3_8_b1.png}
 
\begin{quote}I discovered an old poem in English.\end{quote}

   \includegraphics{Diagrams/3_8_b2.png}

\newpage
  {\setlength{\baselineskip}%
           {0.0\baselineskip}
  \section*{Notes and Instructor Comments}
  \hrulefill \par}
\end{document}

\documentclass[11pt,notitlepage]{article}
\usepackage{graphicx}
\usepackage{amsmath}            % adds more math symbols
\usepackage{amssymb}
\usepackage[usenames,dvipsnames]{color}
\usepackage{multicol}
\usepackage{listings}
\title{LING-300: Homework 1}
\author{Leo Przybylski\\
\texttt{przybyls@arizona.edu}}

\newcommand{\question}[2]{\textbf{#1.} #2}
\newcommand{\subquestion}[2]{\par\hspace{0.5cm} \textbf{#1)} #2}
\newenvironment{answer}{\endpar%

}

    % Give wider margins; gives more text per page.

\setlength{\topmargin}{0.00in}
\setlength{\textheight}{8.75in}
\setlength{\textwidth}{6.625in}
\setlength{\oddsidemargin}{0.0in}
\setlength{\evensidemargin}{0.0in}

\setlength{\parindent}{0.0cm}	% Don't indent the paragraphs
%\setlength{\parskip}{0.4cm}	% distance between paragraphs

\definecolor{ubergray}{RGB}{245,245,245}
\begin{document}
  \maketitle
  {\setlength{\baselineskip}%
           {0.0\baselineskip}
  \section*{Carnie Chapter 4}
  \hrulefill \par}

\question{1}{TREES}
\begin{quote}
[Application of Skills; Basic to Intermediate]
Using the rules we developed in chapter 3, draw the trees for the
following sentences. Many of the sentences are ambiguous. For those
sentences draw one possible tree, indicating the meaning by providing
a paraphrase.
\end{quote}


\subquestion{f}{The book of poems with the bright red cover stinks.}

\includegraphics[width=\textwidth]{Diagrams/4_1_f.png}

\subquestion{g}{Louis hinted Mary stole the purse deftly.}

\includegraphics[width=\textwidth]{Diagrams/4_1_g.png}

\subquestion{h}{The extremely tired students hated syntactic trees with a passion.}

\includegraphics[width=\textwidth]{Diagrams/4_1_h.png}


\question{13}{STRUCTURAL RELATIONS I9 [Application of Skills; Advanced] Consider the following tree:}

\subquestion{1}{What node(s) dominate $N_3$ \emph{grocer}?}

\begin{quote}
  TP, VP, PP, NP
\end{quote}

\subquestion{2}{What node(s) immediately dominate $D_3$ \emph{the}?}

\begin{quote}
  $NP_3$
\end{quote}

\subquestion{3}{Do T \emph{will} and V \emph{buy} form a constituent?}

\begin{quote}
  yes
\end{quote}

\subquestion{4}{What nodes does $N_1$ \emph{bully} c-command?}

\begin{quote}
  $D_1$, $N_1$
\end{quote}

\subquestion{5}{What nodes does $NP_1$ \emph{the big bully}
  c-command?}

\begin{quote}
VP
\end{quote}

\subquestion{6}{What is V \emph{buy}’s mother?}

\begin{quote}
  VP
\end{quote}

\subquestion{7}{What nodes does T \emph{will} precede?}

\begin{quote}
VP
\end{quote}

\subquestion{8}{List all the sets of sisters in the tree.}

\begin{quote}
NP, T, VP

D, AdjP, N

V, NP, PP

P, NP

D, N
\end{quote}

\subquestion{9}{What is the PP’s mother?}

\begin{quote}
  VP
\end{quote}

\subquestion{10}{Do $NP_1$ and VP asymmetrically or symmetrically c-command one
another?}

\begin{quote}
$NP_1$ asymmetrically  c-commands VP.
\end{quote}

\subquestion{11}{List all the nodes c-commanded by V.}

\begin{quote}
  D, N, AdjP, Adj
\end{quote}

\subquestion{12}{What is the subject of the sentence?}

\begin{quote}
  The big bully
\end{quote}

\subquestion{13}{What is the object of the sentence?}

\begin{quote}
  the grocer
\end{quote}

\subquestion{14}{What is the object of the preposition?}

\begin{quote}
  apples
\end{quote}

\subquestion{15}{Is $NP_3$ a constituent of VP?}

\begin{quote}
  Yes.
\end{quote}


\subquestion{16}{What node(s) is $NP_3$ an immediate constituent of?}

\begin{quote}
  $NP_3$ is an immediate constituent of PP
\end{quote}

\subquestion{17}{What node(s) does VP exhaustively dominate?}

\begin{quote}
  V, $NP_2$, $N_2$, PP, P, $NP_3$ $N_3$, $D_3$
\end{quote}

\subquestion{18}{What is the root node?}

\begin{quote}
  TP
\end{quote}

\subquestion{19}{List all the terminal nodes.}

\begin{quote}
$D_1$, 
\end{quote}

\subquestion{20}{What immediately precedes $N_3$ grocer?}

\begin{quote}
$D_1$, Adj, $N_1$, T, V, $N_2$, P, $D_3$, $N_3$ 
\end{quote}

\question{16}{HIAKI}
\begin{quote}
[Data Analysis; Intermediate]
Consider the data from the following sentences of Hiaki (also known as
Yaqui), an Uto-Aztecan language from Arizona and Mexico. Data have
been simplified. (Data from Dedrick and Casad 1999.)
\end{quote}

\subquestion{a}{Tékil né-u ’aáyu-k.
work me-for is
“There is work for me.” (literally: “Work is for me.”)}

\subquestion{b}{Hunáa’a yá’uraa hunáka’a hámutta nokriak. that chief
  that woman defend “That chief defended that woman.”}

\subquestion{Taáwe tótoi’asó’olam káamomólim híba-tu’ure. Hawk chickens young like
“(The) hawk likes young chickens.”}

\subquestion{d}{Tá’abwikasu ’áma yépsak.
different-person there arrived
“A different person arrived there.” (assume there is an adverb not a N)
Assume the rules AdjP  Adj and AdvP  Adj and answer the following
questions.}

\begin{enumerate}
\item What is the NP rule for Hiaki?

\begin{quote}
  NP $\rightarrow$ (D) N (AdjP+) (AdvP)
\end{quote}

\item Do you need a PP rule for Hiaki? Why or why not?
  
  \begin{quote}
    No. Prepositions are treated as verbs, so an NP can have an AdvP
    replacing what is normally the NP dominated by a PP.
  \end{quote}

\item What is the VP rule for Hiaki?

  \begin{quote}
    VP $\rightarrow$ V
  \end{quote}

\item What is the TP rule for Hiaki?
  
  \begin{quote}
    TP $\rightarrow$ NP VP
  \end{quote}

\item Using the rules you developed in questions 1–4, draw the tree for
sentences (b, c, d).

\includegraphics{Diagrams/4_16_b.png}

\includegraphics{Diagrams/4_16_c.png}

\includegraphics{Diagrams/4_16_d.png}

\item What is the subject of sentence (b)?

\begin{quote}
  \emph{That chief} is the subject.
\end{quote}

\item Is there an object in (d)? If so, what is it?

  \begin{quote}
    \emph{There} is the object.
  \end{quote}

\item What node(s) does hunáa’a c-command in (b)?
  
  \begin{quote}
    D \emph{that} and N \emph{woman}
  \end{quote}

\item What node(s) does hunáa’a yá’uraa c-command in (b)?

  \begin{quote}
    D \emph{that} and N \emph{woman}
  \end{quote}

\item What does ’áma precede in (d)?

\begin{quote}
’áma precedes yépsak.
\end{quote}

\item What node immediately dominates káamomólim in (c)?

\begin{quote}
  NP dominates AdjP \emph{káamomólim}.
  \end{quote}

\item What nodes dominate káamomólim in (c)?

\begin{quote}
  TP and NP dominates AdjP \emph{káamomólim}.
  \end{quote}

\item What node immediately precedes káamomólim in (c)?

\begin{quote}
  N \emph{tótoi’asó’olam} immediately precedes káamomólim.
\end{quote}

\item What nodes precede káamomólim in (c)?

  \begin{quote}
   N \emph{Taáwe} and  N \emph{tótoi’asó’olam} precede káamomólim.
  \end{quote}

\item Does káamomólim c-command táawe in (c)?
  
  \begin{quote}
    Yes. It does.
  \end{quote}

\item Do hunáka’a and hámutta symmetrically c-command one another in
(b)?

\begin{quote}
  Yes, they do.
\end{quote}


\end{enumerate}

  {\setlength{\baselineskip}%
           {0.0\baselineskip}
  \section*{Carnie Chapter 5}
  \hrulefill \par}

\question{4}{BINDING PRINCIPLES}
\begin{quote}
[Application of Skills, Data Analysis; Intermediate]
Explain why the following sentences are ungrammatical. For each
sentence, say what the binding domain of the NP causing the problem
is, if it is c- commanded by its binder (antecedent), and name the
binding condition that is violated.
\end{quote}

\subquestion{a}{*$Michael_i$ loves $him_i$.}

\begin{quote}
  Michael is the antecedent of him and they c-command each other. The Binding Domain is
  \emph{Michael loves him}. This violates Principle B because
  \emph{him} is no longer free within the binding domain.
\end{quote}

\subquestion{b}{*Hei loves Michaeli.}

\begin{quote}
  Michael is the antecedent of he and they c-command each other. The Binding Domain is
  \emph{He loves Michael}. This violates Principle B because
  \emph{he} is no longer free within the binding domain.
\end{quote}

\subquestion{c}{*$Michael_i$’s $father_j$ loves $himself_i$.}

\begin{quote}
  Michael is the antecedent of himself. The Binding Domain is
  \emph{Michael's father loves himself}. This violates locality
  constraint because \emph{himself} is local to \emph{father} instead
  of its antecedent \emph{Michael}.
\end{quote}

\subquestion{d}{*Michaeli’s fatherj loves himj.}

\begin{quote}
  father is the antecedent of him. The Binding Domain is
  \emph{Michael's father loves him}. This violates Principle B
   because \emph{him} is not free within the binding domain.
\end{quote}

\subquestion{e}{*Susani thinks that John should marry herselfi.}

\begin{quote}
  Susan is the antecedent of herself. The Binding Domain is
  \emph{John should marry herself}. This violates Binding Principle A 
  because \emph{herself} is not bound within the binding domain.
\end{quote}

\subquestion{f}{*John thinks that Susani should kiss heri.}

\begin{quote}
  Susan is the antecedent of her. The Binding Domain is
  \emph{Susan should kiss her}. This violates Principle B
   because \emph{her} is not free within the binding domain.
\end{quote}

\question{CHALLENGE PROBLEM SET 1}{WH-QUESTIONS}
\begin{quote}
[Critical Thinking; Challenge]
What problem(s) does the following sentence make for the binding theory as we have sketched it in this chapter? Can you think of a solution? (Hint: con- sider the non-question form of this sentence John despises these pictures of himself.)
Which pictures of himselfi does Johni despise? Assume the following
tree for this sentence:
\end{quote}

\begin{quote}
  The non-question form of the sentence violates the locality
  constraint because \emph{himself} is not near the antecedent
  \emph{John}. Also, due to the complementizer \emph{does},
  \emph{himself} and \emph{John} are not in the same binding
  domain. This then violates Principle A. Still, this sentence seems
  grammatical to me.

  The rule could be changed to target clauses or binding domains with
  more than one possible antecedent.
\end{quote}

\newpage
  {\setlength{\baselineskip}%
           {0.0\baselineskip}
  \section*{Notes and Instructor Comments}
  \hrulefill \par}
\end{document}

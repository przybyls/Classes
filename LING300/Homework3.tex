\documentclass[11pt,notitlepage]{article}
\usepackage{graphicx}
\usepackage{amsmath}            % adds more math symbols
\usepackage{amssymb}
\usepackage[usenames,dvipsnames]{color}
\usepackage{multicol}
\usepackage{listings}
\title{LING-300: Homework 3}
\author{Leo Przybylski\\
\texttt{przybyls@arizona.edu}}

\newcommand{\question}[2]{\textbf{#1.} #2}
\newcommand{\subquestion}[2]{\par\hspace{0.5cm} \textbf{#1)} #2}
\newenvironment{answer}{\endpar%

}

    % Give wider margins; gives more text per page.

\setlength{\topmargin}{0.00in}
\setlength{\textheight}{8.75in}
\setlength{\textwidth}{6.625in}
\setlength{\oddsidemargin}{0.0in}
\setlength{\evensidemargin}{0.0in}

\setlength{\parindent}{0.0cm}	% Don't indent the paragraphs
%\setlength{\parskip}{0.4cm}	% distance between paragraphs

\definecolor{ubergray}{RGB}{245,245,245}
\begin{document}
  \maketitle
  {\setlength{\baselineskip}%
           {0.0\baselineskip}
  \section*{Carnie Chapter 3}
  \hrulefill \par}
\question{1}{TREES: NPS, ADJPS AND ADVPS}
\begin{quote}
  [Application of Skills; Basic]
  Draw the trees for the following AdjPs, AdvPs, and NPs: 
\end{quote}  

\subquestion{a}{very smelly}

\begin{quote}
\includegraphics{Diagrams/3_1_a.png}
\end{quote}

\subquestion{b}{too quickly}

\begin{quote}
\includegraphics{Diagrams/3_1_b.png}
\end{quote}

\subquestion{c}{much too quickly}

\begin{quote}
\includegraphics{Diagrams/3_1_c.png}
\end{quote}


\subquestion{d}{very much too quickly}

\begin{quote}
\includegraphics{Diagrams/3_1_d.png}
\end{quote}

\subquestion{e}{the old shoelace}

\begin{quote}
\includegraphics{Diagrams/3_1_e.png}
\end{quote}

\subquestion{g}{these very finicky children}

\begin{quote}
\includegraphics{Diagrams/3_1_g.png}
\end{quote}

\question{4}{ENGLISH}
\begin{quote}
[Application of Skills and Knowledge; Basic to Advanced]
Draw phrase structure trees and bracketed diagrams for each of the
follow- ing sentences, indicate all the categories (phrase (e.g., NP)
and word level (e.g., N)) on the tree. Use the rules given above in
the “Ideas” summary of this chapter. Be careful that items that modify
one another are part of the same constituent. Treat words like can,
should, might, was, as instances of the category T (tense). (Sentences
d–h are from Sheila Dooley.)
\end{quote}

\subquestion{a}{The kangaroo hopped over the truck.}

\begin{quote}
 [$_{TP}$[$_{NP}$The kangaroo] [$_{VP}$hopped [$_{PP}$over [$_{NP}$the truck]]]].

\includegraphics{Diagrams/3_4_a.png}
\end{quote}

\subquestion{b}{I haven’t seen this sentence before. [before is a P,
  haven’t is a T]}

\begin{quote}
  [$_{TP}$[$_{NP}$I] [$_{T}$haven't] [$_{VP}$seen [$_{NP}$this sentence] [$_{AdvP}$before]]].

\includegraphics{Diagrams/3_4_b.png}
\end{quote}

\subquestion{c}{Susan will never sing at weddings. [never is an Adv]}

\begin{quote}
[$_{TP}$[$_{NP}$Susan] [$_{T}$will] [$_{VP}$[$_{AdvP}$never] sing [$_{PP}$at weddings]]].

\includegraphics{Diagrams/3_4_c.png}
\end{quote}

\subquestion{d}{The officer carefully inspected the license.}

\begin{quote}
[$_{TP}$[$_{NP}$The officer] [$_{VP}$[$_{Advp}$carefully] inspected [$_{NP}$the license]]].

\includegraphics{Diagrams/3_4_d.png}
\end{quote}


\question{6}{HIXKARYANA}
\begin{quote}
[Application of Skills; Basic/Intermediate]
Look carefully at the following data from a Carib language from Brazil
(the glosses have been slightly simplified from the original). In your
analysis do not break apart words. (Data from Derbyshire 1985.)
\end{quote}

\begin{enumerate}
\item Kuraha yonyhoryeno biyekomo. bow made boy “The boy
  made a bow.”
\item Newehyatxhe woriskomo komo. take-bath women all “All
  the women take a bath.”
\item Toto heno komo yonoye kamara. person dead all ate jaguar “The jaguar ate all the dead people.”
\end{enumerate}

Now answer the following questions about Hixkaryana:

\subquestion{1}{Is there any evidence for a determiner category in
  Hixkaryana? Be sure to consider quantifier words as possible
  determiners (like some and all).}
Yes, all is used to direct context toward a non-descript set of things.

\subquestion{2}{Posit an NP rule to account for Hixkaryana. (Be careful to do it
  for the second line, the word-by-word gloss, in these examples not
  the third line.) Assume there is an AdjP rule: AdjP  Adj.}

\begin{quote}
  NP$\rightarrow$N (Adj+) (D)
\end{quote}

\subquestion{3}{Posit a VP rule for Hixkaryana.}

\begin{quote}
VP$\rightarrow$ (NP) V
\end{quote}

\subquestion{4}{Posit a TP rule for Hixkaryana.}

\begin{quote}
  TP$\rightarrow$ VP NP
\end{quote}

\subquestion{5}{What is the part of speech of newehyatxhe? How do you
  know?}

\begin{quote}
  verb phrase. It is the word for taking a bath. 
\end{quote}

\subquestion{6}{Draw the trees for (a) and (c) using the rules you posited above. (Hint: if
your trees don’t work, then you have probably made a mistake in the
rules.)}

[$_{TP}$[$_{VP}$[$_{NP}$Kuraha] yonyhoryeno] [$_{NP}$biyekomo]]

\begin{quote}
  \includegraphics{Diagrams/3_6_61.png}
\end{quote}

[$_{TP}$[$_{VP}$[$_{NP}$Toto [$_{AdjP}$heno] [$_{AdjP}$komo]] yonoye] [$_{NP}$kamara]]

\begin{quote}
\includegraphics{Diagrams/3_6_62.png}
\end{quote}

\subquestion{7}{Give bracketed diagrams for the same sentences.}

\question{9}{STRUCTURE}
\begin{quote}
[Application of Knowledge; Intermediate]
In the following sentences a sequence of words is marked as a
constituent with square brackets. State whether or not it is a real
constituent, and what criteria (that is constituency tests) you
applied to determine that result.
\end{quote}

\subquestion{a}{Susanne gave [the minivan to Petunia].}

\begin{quote}
  ``the minivan to Petunia'' is not a constituent. It fails the
  replacement, movement, and standalone constituency tests. It is
  actually 2 constituents.
\end{quote}

\subquestion{b}{Clyde got [a passionate love letter from Stacy].}

\begin{quote}
  ``a passionate love letter from Stacy'' is not a constituent. It passes
  the replacement test, but not the standalone or movement tests. Like
  a, this is also more like 2 constituents than 1.
\end{quote}

\newpage
  {\setlength{\baselineskip}%
           {0.0\baselineskip}
  \section*{Notes and Instructor Comments}
  \hrulefill \par}
\end{document}

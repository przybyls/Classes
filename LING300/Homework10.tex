\documentclass[11pt,notitlepage]{article}
\usepackage{graphicx}
\usepackage{amsmath}            % adds more math symbols
\usepackage{amssymb}
\usepackage[usenames,dvipsnames]{color}
\usepackage{multicol}
\usepackage{listings}
\title{LING-300: Homework 10}
\author{Leo Przybylski\\
\texttt{przybyls@arizona.edu}}

\newcommand{\question}[2]{\textbf{#1.} #2}
\newcommand{\subquestion}[2]{\par\hspace{0.5cm} \textbf{#1)} #2}
\newenvironment{answer}{\endpar%

}

    % Give wider margins; gives more text per page.

\setlength{\topmargin}{0.00in}
\setlength{\textheight}{8.75in}
\setlength{\textwidth}{6.625in}
\setlength{\oddsidemargin}{0.0in}
\setlength{\evensidemargin}{0.0in}

\setlength{\parindent}{0.0cm}	% Don't indent the paragraphs
%\setlength{\parskip}{0.4cm}	% distance between paragraphs

\definecolor{ubergray}{RGB}{245,245,245}
\begin{document}
  \maketitle
  {\setlength{\baselineskip}%
           {0.0\baselineskip}
  \section*{Carnie Chapter 9}
  \hrulefill \par}

\question{1}{ITALIAN}
\begin{quote}
[Data Analysis: Basic]
Consider the following data from Italian. Assume non is like French
ne- and is irrelevant to the discussion. Concentrate instead on the
positioning of the word più, ‘anymore.’ (Data from Belletti 1994.)
\end{quote}
\subquestion{a}{Gianni non ha più parlato.}
Gianni non has anymore spoken “Gianni does not speak anymore.”

\subquestion{b}{Gianni non parla più.}
Gianni non speaks anymore “Gianni speaks no more.”
On the basis of this very limited data, is Italian a verb raising
language or an affix lowering language?

\question{2}{HAITIAN CREOLE VERB PLACEMENT}
[Data Analysis; Basic]
Consider the following sentences from Haitian Creole. Is Creole a verb rais- ing language or an affix lowering language? Explain your answer. (Data from DeGraff 2005.)
a) Bouki deja konnen Boukinèt Bouki already knows Boukinèt “Bouki already knows Boukinet.”
b) Bouki pa konnen Boukinèt Bouki neg knows Boukinèt “Bouki doesn’t
know Boukinèt.”

\question{6}{AMERICAN VS. BRITISH ENGLISH VERB HAVE}
\begin{quote}
[Critical Thinking; Basic/Intermediate]
English has two verbs to have. One is an auxiliary seen in sentences
like (a):
\end{quote}

a) I have never seen this movie.
The other indicates possession:
b) I never have a pen when I need it.
You will note from the position of the adverb never that the possessive verb have is a main verb, whereas the auxiliary have is raises to T.
Part 1: Consider the following data from American English. How does it sup- port the idea that auxiliary have ends up in T, but possessive have is a main verb, and stays downstairs (i.e., has affix lowering applied)?
c) I have had a horrible day.
d) I have never had a pencil case like that!
e) Have you seen my backpack?
f) *Have you a pencil?
Part 2: Consider now the following sentence, which is grammatical in some varieties of British English:
g) Have you a pencil?
Does the possessive verb have in these dialects undergo V → T
movement? How can you tell?

8. ENGLISH5
[Data Analysis; Intermediate]
Consider the italicized noun phrases in the following sentences:
a) I ate something spicy.
b) Someone tall was looking for you.
c) I don’t like anyone smart.
d) I will read anything interesting.
One analysis that has been proposed for noun phrases like the ones above involves generating elements like some and any as determiners, and gener- ating elements one and thing as nouns (under N), and then doing head-to- head movement of the Ns up to D. The tree below illustrates this analysis:
DP
D'
D NP
N'
N
AP
N'
Give an argument in favor of this analysis, based on the order of
elements within the noun phrase in general, and the order of elements
in the noun phrases above.


9. ENGLISH TREES
[Application of Skills; Basic to Advanced]
Draw trees for the following English sentences; be sure to indicate all trans- formations with arrows.
a) I have always loved peanut butter.
b) I do not love peanut butter.
c) Martha often thinks Kim hates phonology.
d) Do you like peanut butter?
e) Have you always hated peanut butter?
f) Are you always so obtuse? (Assume that AdjP can be a complement to
T if it is a predicate, as in this case)
g) Will you bring your spouse?
h) Has the food been eaten?
i) Mike is always eating peanuts.

\newpage
  {\setlength{\baselineskip}%
           {0.0\baselineskip}
  \section*{Notes and Instructor Comments}
  \hrulefill \par}
\end{document}
